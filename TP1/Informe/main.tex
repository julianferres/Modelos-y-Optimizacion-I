\documentclass[a4paper]{article}
\usepackage[spanish]{babel}
\usepackage[utf8x]{inputenc}
\usepackage{amsmath}
\usepackage{changepage}
\usepackage{amsfonts}
\usepackage{mathtools}
\usepackage{multirow}
\usepackage{graphicx}
\usepackage{amssymb}
\usepackage{listings}
\usepackage{setspace}
\usepackage[left=20mm, right=20mm,top=20mm]{geometry}
\usepackage[authoryear,round]{natbib}
\usepackage[automark]{scrpage2}
\usepackage{subcaption}
\usepackage{float}
\usepackage{enumerate}
\usepackage{geometry}
\usepackage{xcolor}
\usepackage{caption}

\graphicspath{ {graficos/} }

\begin{document}


\thispagestyle{empty}
\title{TP1 - Modelos y Optimización I}

		\begin{figure}[H]
	\centering
    \includegraphics[width=0.15\textwidth]{logo.jpg}
   \end{figure}


	\begin{center}
		\huge{\textsc{Universidad de Buenos Aires}}\\
		\huge{\textsc{Facultad de Ingeniería}}\\
		\vspace*{3mm}
		\large{ 2\textsuperscript{do} Cuatrimestre 2019}
	\end{center}

	\vspace*{20mm}

	\begin{center}
		\huge{\underline{\textsc{Modelos y Optimización I (71.14)}}}\\
	\end{center}



	\begin{large}

	\begin{tabbing}
		\hspace{34mm} \= \+
		\textsc{\bf{\Large{Informe Trabajo Práctico 1}}}\\
		\textsc{Fecha:} {\today}\\
	\end{tabbing}

\vspace{20mm}

    \begin{tabbing}
    	\hspace{20mm} \= \+
        \textsc{Integrantes:}\\
    \end{tabbing}

	\end{large}

\vspace{-7mm}

\begin{center}
\begin{tabular}{|c|c|c|}
	\hline
	\textbf{Apellido y Nombre} & \textbf{Padrón} & \textbf{Correo Electrónico}  \\

    \hline
     Ferres, Julian &101483 & julianferres@gmail.com \\
    \hline
    ---- & ----- & cyntgamarra@gmail.com\\
	\hline
    Loguercio, Sebastian & 100517 & seba21log@gmail.com  \\
    \hline


 \end{tabular}
\end{center}

\newpage
\tableofcontents
\thispagestyle{empty}


\newpage
\setcounter{page}{1}

\section{Parte A}

\subsection{Análisis}

\subsection{Objetivo}
Alaska y Hawaii
\subsection{Hipótesis}
Las hipotesis impuestas son las siguientes:

\begin{itemize}
	\item La heladera dura todo el viaje, por lo que habrá que comprar solo una.
\end{itemize}


\subsection{Modelo de programación lineal continua}
\subsection{Constantes}

\begin{itemize}
	\item $d_{ij}$: Distancia entre la ciudad $i$ y la ciudad $j$ (en km).
	%Precio nafta


\end{itemize}

\subsubsection{Variables}
\begin{itemize}
	\item $U_{i}$ Orden en el que la capital i fue visitada.

	\item $Y_{ij}$ El camino entre la capital $i$ y la $j$ está en el tour.
	\item $D$: Distancia recorrida (en km).
	% Ver como definir las variables del costo diferido en la carpeta
	\item $DES_j$: Descansan 2 dias en la ciudad $j$
	%Ver si lo queremos entre cada ij o en funcion de D --> \item $E_{ij}$: Cantidad de veces que se detienen a estirarse
	\item $H$: Compro la heladera (binaria).
	\item $Agua$: Precio de una botella de agua (usd), entera $Agua \in \{2,3\}$

\end{itemize}

\subsubsection{Ecucaciones}

$ D = \sum_{i}\sum_{j} d_{ij} Y_{ij}$


\subsection{Modelo en computadora}
\subsection{Resolución}

\subsection{Análisis de la solución}

\newpage
\section{Parte B}

\subsection{Análisis}

\subsection{Objetivo}
Determinar la cantidad de pases Gold, pases Silver y paquetes de merchandising a vender para obtener la mayor ganancia posible en un periodo determinado.

\subsection{Hipótesis}

\begin{itemize}
    \item Todos los pases se venden.
    \item Las entradas de protocolo son Silver o Gold o cualquiera??
    \item Dice ``un par'' de entradas de protocolo. Osea 2?
    \item Los 100 pases Gold al dueño del predio generan ganancia?

\end{itemize}


\subsection{Modelo de programación lineal continua}
\subsubsection{Variables}

$M_{gold} =$ Cantidad de metros cuadrados dedicados a ubicaciones Gold.\\

$M_{silv} =$ Cantidad de metros cuadrados dedicados a ubicaciones Silver.\\

$S_{gold} =$ Cantidad de pases Gold a vender.\\

$S_{silv} =$ Cantidad de pases Silver a vender.\\

$P =$ Cantidad de paquetes de merchandisign comprados.\\

$P_{gold} =$ Cantidad de paquetes comprados destinados a ubicaciones Gold.\\

$P_{silv} =$ Cantidad de paquetes comprados destinados a ubicaciones Silver.\\


\subsubsection{Ecuaciones}

\begin{itemize}
    \item 8000 metros cuadrados en total disponibles:
    $$ M_{gold} + M_{silv}  \leq 8000\,m²$$

    \item Relación pases Gold y metros cuadrados:
    $$ S_{gold} = \tfrac{1\: pase}{m²} \cdot M_{gold} $$

    \item Relación pases Silver y metros cuadrados:
    $$ S_{silv} = \tfrac{2\: pases}{m²} \cdot M_{silv} $$

    \item 100 pases Golds al dueño del predio: (se cobran ??)
    $$ S_{gold} \geq 100\,pases $$

    \item 500 pases Silver vendidos para pagar la reserva del predio:
    $$ S_{silv} \geq 500\: pases$$

    \item Paquetes de merchandising:
    $$ P = P_{gold} + P_{silv} $$

    \item Se pueden comprar hasta 800 paquetes:
    $$ P \leq 800\: paquetes $$

    \item Un paquete cubre hasta 20 personas que compraron un pase Silver:
    $$ P_{silv} \leq \tfrac{1\: paquete} {20\: pases} \cdot S_{silv}$$

    \item Un paquete cubre hasta 8 personas que compraron un pase Gold:
    $$ P_{gold} \leq \tfrac{1\: paquete} {8\: pases} \cdot S_{gold}$$


\end{itemize}

\subsubsection{Funcional}


\subsection{Modelo en computadora}

\subsection{Resolución}

\subsection{Análisis de la solución}



\end{document}
